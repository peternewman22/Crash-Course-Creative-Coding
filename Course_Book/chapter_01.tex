\documentclass{aip-book}

% The packages loaded in this preamble are used only for the elements of the example chapter.
% If you have no need of a certain package in your chapter, 
% or would like to use a different solution, feel free to remove them.

% Miscellaneous
\usepackage{graphicx,hologo,mwe,url} % for elements of the example chapter
\usepackage{array,booktabs,threeparttable} % useful for the table example
\usepackage{harvard} % for the bibliography of the example
\newcommand*\cmd[1]{\texttt{#1}} % for documenting LaTeX commands and environments

% Math symbols and environments
\usepackage{amsmath}
\newcommand{\vect}[1]{\mathbf{#1}}

% For theorems, definitions, and proofs
\usepackage{amsthm}
\newtheorem{theorem}{Theorem}
\theoremstyle{definition}
\newtheorem{definition}{Definition}

\begin{document}
\setchapter{4} % set this to your chapter number (this will be provided during final typesetting)
\chapter{Contribution Title} % the title of your chapter

% each author's information is entered with the \chapterauthor macro
% this section is only needed for "contributed" books
% for "authored" books, comment out or remove the \chapterauthor commands
\chapterauthor{Name of First Author}{Name of Institute}{Address of Institute}{name@email.com}
\chapterauthor{Name of Second Author}{Name of Institute}{Address of Institute}{name@email.address}

\begin{abstract}
Each chapter should be preceded by an abstract (10–15 lines long) that summarizes the content. 
The abstract will appear online on Scitation and be available with unrestricted access. 
This allows unregistered users to read the abstract as a teaser for the complete chapter. 
\end{abstract}

\keywords{A list up to 10 key words to enhance the chapter’s searchability}

\section{Introduction}

This template may be used to contribute chapters for AIP Publishing Books. 
Please note that this template shows only an approximation of the page layout and document elements.
This is to assist you with preparing your contribution for final typesetting.
For example, planning line breaks for long equations, or sizes for figure and table layouts.
Do not worry too much about final figure and table placement.
This will be finalized at the production typesetting phase, where you'll have the opportunity to review.

\section{Section Heading}\label{sec:xref}

Instead of simply listing headings of different levels we recommend to let every heading be followed by at least a short passage of text. 
Further on please use the \LaTeX\ automatism for all your cross-references and citations. 
And please note that the first line of text that follows a heading is not indented, whereas the first lines of all subsequent paragraphs are.

\section{Section Heading}

Use the standard \cmd{equation} environment to typeset your equations, e.g.
\begin{equation}
    a \times b = c,
\end{equation}
however, for multiline equations we recommend to use the \cmd{align} environment from \cmd{amsmath}%
\footnote{In physics texts, please depict your vectors in \emph{\emph{boldface-italic}} type -- as is customary for a wide range of physical subjects.}.
\begin{align}
a \times b &= c \notag \\
\vect{a} \cdot \vect{b} &= \vect{c}.
\end{align}

\subsection{Subsection Heading}\label{sec:headings}

Instead of simply listing headings of different levels we recommend to let every heading be followed by at least a short passage of text. 
Further on please use the \LaTeX\ automatism for all your cross-references and citations as has already been described in Sect.~\ref{sec:xref}.

\begin{quotation}
Please do not use quotation marks when quoting texts! Simply use the \cmd{quotation} environment.
\end{quotation}

\subsubsection{Subsubsection Heading}

Instead of simply listing headings of different levels we recommend to let every heading be followed by at least a short passage of text. 
Further on please use the \LaTeX\ automatism for all your cross-references and citations as has already been described in Sect.~\ref{sec:headings}, see also Fig.~\ref{fig:sidecap}%
\footnote{If you copy text passages, figures, or tables from other works, you must obtain permission from the copyright holder (usually the original publisher). 
Please enclose the signed permission with the manuscript. The sources must be acknowledged either in the captions, as footnotes or in a separate section of the book.}.
Do not worry too much about figure and table placement in your manuscript.
This will be finalized at the production typesetting phase.
\begin{figure}[h]
    \centering
    \fcapside[\FBwidth]{\caption{
      If the width of the figure is less than 7.8 cm, use the \cmd{fcapside} command to flush the caption on the left side of the page.
      The \cmd{aip-book} class uses the \cmd{floatrow} package; you can use its commands to set up sub-figure layouts and similar if needed.}
    \label{fig:sidecap}}%
    {\includegraphics[width=4cm]{example-image-a}}
\end{figure}

Please note that the first line of text that follows a heading is not indented, whereas the first lines of all subsequent paragraphs are.

\paragraph{Paragraph Heading}

Instead of simply listing headings of different levels we recommend to let every heading be followed by at least a short passage of text. 
Further on please use the \LaTeX\ automatism for all your cross-references and citations as has already been described in Sect.~\ref{sec:xref}.

Please note that the first line of text that follows a heading is not indented, whereas the first lines of all subsequent paragraphs are.

For typesetting numbered lists, we recommend to use the \verb+enumerate+ environment 
\begin{enumerate}
  \item Livelihood and survival mobility are oftentimes outcomes of uneven socioeconomic development.
  \begin{enumerate}
    \item Livelihood and survival mobility are oftentimes outcomes of uneven socioeconomic development.
    \item Livelihood and survival mobility are oftentimes outcomes of uneven socioeconomic development.
  \end{enumerate}
  \item Livelihood and survival mobility are oftentimes outcomes of uneven socioeconomic development.
\end{enumerate}

\subparagraph{Subparagraph Heading}

In order to avoid simply listing headings of different levels we recommend to let every heading be followed by at least a short passage of text. 
Use the \LaTeX\ automatism for all your cross-references and citations as has already been described in Sect.~\ref{sec:headings}, see also Fig.~\ref{fig:example}.
\begin{figure}
    \centering
    \includegraphics[width=8cm]{example-image-b}
    \caption{
      If the width of the figure is greater than 7.8 cm, use the standard caption arrangement.}
    \label{fig:example}
\end{figure}

For unnumbered list we recommend to use the \verb+itemize+ environment:
\begin{itemize}
  \item Livelihood and survival mobility are oftentimes outcomes of uneven socioeconomic development, cf. Table~\ref{tab:mRNA}.
  \begin{itemize}
      \item Livelihood and survival mobility are oftentimes outcomes of uneven socioeconomic development.
      \item Livelihood and survival mobility are oftentimes outcomes of uneven socioeconomic development.
  \end{itemize}
  \item Livelihood and survival mobility are oftentimes outcomes of uneven socioeconomic development.
\end{itemize}

\begin{table}
\centering
\begin{threeparttable}[t]
\caption{Please write your table caption here.}\label{tab:mRNA}
\begin{tabular}{l>{\raggedright}p{1.5cm}l>{\raggedright\arraybackslash}p{4cm}}
  \toprule
  Classes & Subclass & Length & Action Mechanism \\
  \midrule
  Translation & mRNA\tnote{a} & 22 (19--25) & Translation repression, mRNA cleavage \\
  Translation & mRNA cleavage & 21 & mRNA cleavage \\
  Translation & mRNA & 21--22 & mRNA cleavage \\
  Translation & mRNA & 24--26 & Histone and DNA Modification \\
  \bottomrule
\end{tabular}
\begin{tablenotes}
  \item[a] Table footnote (with superscript)
\end{tablenotes}
\end{threeparttable}
\end{table}

\runinbold{Run-in Heading Boldface Version}
Use the \LaTeX\ automatism for all your cross-references and citations as has already been described in Sect.~\ref{sec:headings}.

\runinitalic{Run-in Heading Italic Version}  
Use the \LaTeX\ automatism for all your cross-references and citations as has already been described in Sect.~\ref{sec:headings}.

\section{Section Heading}

Instead of simply listing headings of different levels we recommend to let every heading be followed by at least a short passage of text. 
Further on please use the \LaTeX\ automatism for all your cross-references and citations as has already been described in Sect.~\ref{sec:headings}.

Please note that the first line of text that follows a heading is not indented, whereas the first lines of all subsequent paragraphs are.

\begin{description}
  \item[Type 1] That addresses central themes pertaining to migration, health, and disease. 
  In Sect.~\ref{sec:xref}, Wilson discusses the role of human migration in infectious disease distributions and patterns.
  \item[Type 2] That addresses central themes pertaining to migration, health, and disease. 
  In Sect.~\ref{sec:headings}, Wilson discusses the role of human migration in infectious disease distributions and patterns.
\end{description}

\subsection{Subsection Heading}

In order to avoid simply listing headings of different levels we recommend to let every heading be followed by at least a short passage of text. 
Use the \LaTeX\ automatism for all your cross-references and citations citations as has already been described in Sect.~\ref{sec:headings}.

Please note that the first line of text that follows a heading is not indented, whereas the first lines of all subsequent paragraphs are.

\subsubsection{Subsubsection Heading}

Instead of simply listing headings of different levels we recommend to let every heading be followed by at least a short passage of text. 
Further on please use the \LaTeX\ automatism for all your cross-references and citations as has already been described in Sect.~\ref{sec:headings}.

Please note that the first line of text that follows a heading is not indented, whereas the first lines of all subsequent paragraphs are.

\begin{theorem}
Theorem text goes here.
\end{theorem}

\begin{definition}
Definition text goes here.
\end{definition}

\begin{proof}
Proof text goes here.
\end{proof}

\paragraph{Paragraph Heading}

Instead of simply listing headings of different levels we recommend to let every heading be followed by at least a short passage of text. 
Further on please use the \LaTeX\ automatism for all your cross-references and citations as has already been described in Sect.~\ref{sec:headings}.

Note that the first line of text that follows a heading is not indented, whereas the first lines of all subsequent paragraphs are.

\begin{theorem}
Theorem text goes here.
\end{theorem}

\begin{definition}
Definition text goes here.
\end{definition}

\begin{proof}
Proof text goes here.
\end{proof}

\begin{acknowledgement}
If you want to include acknowledgments of assistance and the like at the end of an individual chapter please use the \cmd{acknowledgement} environment.
\end{acknowledgement}

\section*{Appendix}

When placed at the end of a chapter or contribution (as opposed to at the end of the book), the numbering of tables, figures, and equations in the appendix section continues on from that in the main text. 
Hence please \emph{do not} use the \cmd{appendix} command when writing an appendix at the end of your chapter or contribution. 
If there is only one the appendix is designated “Appendix”, or “Appendix 1”, or “Appendix 2”, etc.~if there is more than one.

\section{References}

References must be \emph{cited} in the text by number (preferred).%
\footnote{
  Make sure that all references from the list are cited in the text. 
  Those not cited should be moved to a separate Further Reading section or chapter.
}
The reference list should ideally be \emph{sorted} in numerical order. 

The \emph{styling} of references%
\footnote{
  Always use the standard abbreviation of a journal’s name according to the ISSN List of Title Word Abbreviations, see \url{http://www.issn.org/en/node/344}.} 
depends on the subject of your book but the citations should follow Harvard style.
Here are some example citations for demonstration purposes 
\cite{feyn54,Bire82,Beutler1994,Quinn2001}.

You may use a manual \cmd{thebibliography} environment, \hologo{BibTeX}, or \cmd{biblatex} and \cmd{biber} to prepare your references.
Your reference list and citations should follow Harvard style.
References should appear as an unnumbered section at the end of your chapter.

\bibliographystyle{dcu}
\bibliography{aipsamp}

\end{document}